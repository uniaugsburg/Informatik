\documentclass{article}
\usepackage{amsfonts}
\usepackage{mathtools}
\usepackage[a4paper, total={7.5in, 10.5in}]{geometry}

\title{Informatik 3 - Zusammenfassung}

\begin{document}
	\section*{Gr\"o\ss enordnungen}
		\begin{align*}
			\textbf{$\omega(n)$: } f \in \omega (g) & \xLeftrightarrow{def} g \in o(f) & \textbf{Konstant } & \Theta(1) & \\
			\textbf{$\Omega(n)$: } f \in \Omega (g) & \xLeftrightarrow{def} g \in \mathcal{O}(f) & \textbf{Logarithmisch } & \Theta(\log n)\\
			\textbf{$\Theta (n)$: } f \in \Theta (g) & \xLeftrightarrow{def} f \in \mathcal{O}(g) \wedge f \in \Omega (g) & \textbf{Quasilinear } & \Theta(n)\\
			& \xLeftrightarrow{\ \ \ \ } f \in \mathcal{O}(g) \wedge g \in \mathcal{O}(f) & \textbf{Linear } & \Theta(n)\\
			& \xLeftrightarrow{\ \ \ \ } \mathcal{O}(f) = \mathcal{O}(g) & \textbf{Quadratisch } & \Theta(n^2)\\
			\textbf{$\mathcal{O}(n)$:}\hspace{1.45cm} & \xLeftrightarrow{def} \{f: M \rightarrow M | \exists c,d \in M: f \leq c \cdot n + d\} & \textbf{Kubisch } & \Theta(n^3)\\
			\textbf{$o(n)$: } f \in o(g) & \xLeftrightarrow{def} f \in \mathcal{O}(g) \wedge f \not\in \Omega (g) & \textbf{Polynomiell } & \cup_{k \in \mathbb{N}} \Theta(n^k)\\
			& \xLeftrightarrow{\ \ \ \ } \mathcal{O}(f) \subset \mathcal{O} (g) & \textbf{Exponentiell } & \cup_{k \in \mathbb{N}} \Theta(k^n)\\
			& & \textbf{Faktoriell } & \Theta (n!)\\
		\end{align*}
\end{document}